\documentclass{book}
\usepackage[showframe, inner=0.75in,outer=0.5in, top=0.5in, bottom=0.5in, paperwidth=6in, paperheight=9in]{geometry}
\usepackage{amsmath}

\begin{document}
  \newcommand{\example}{
    \noindent\textbf{Example\getcount:}
  }
  \newcommand{\remark}{
    \noindent\textbf{Remark\getcount:}
  }
  \newcommand{\axiom}{
    \noindent\textbf{Axiom\getcount:}
  }
  \newcommand{\theorem}{
    \noindent\textbf{Theorem\getcount:}
  }
  \newcommand{\theorec}{
    \phantom{\textbf{Theorem:}}
  }
  \newcommand{\lemma}{
    \noindent\textbf{Lemma\getcount:}
  }
  \newcommand{\lemmc}{
    \phantom{\textbf{Lemma:}}
  }
  \newcommand{\idea}{
    \noindent\textbf{idea:}
  }
  \newcommand{\suppose}{
    \textit{Suppose:}
  }
  \newcommand{\supposc}{
    \hphantom{\textit{Suppose:}}
  }
  \newcommand{\infer}{
    \textit{Infer:}
  }
  \newcommand{\infec}{
    \hphantom{\textit{Infer:}}
  }
  \newcommand{\getcount}{
    \arabic{pcount}.\arabic{scount}.\arabic{tcount}
  }
  \newcommand{\ind}{
    \phantom{\texttt{.}}
  }


  \newcounter{pcount}
  \newcounter{scount}
  \newcounter{tcount}

  \makeatletter
  \@addtoreset{scount}{pcount}
  \makeatother

  \makeatletter
  \@addtoreset{tcount}{scount}
  \makeatother

  \stepcounter{pcount}
  \stepcounter{scount}
  \stepcounter{tcount}

  {\axiom Properties of Equivalence \stepcounter{tcount}\\\\
    \noindent The following is assumed to be true for all sets $S$, for all $\alpha, \beta, \text{and}, \gamma$ in $S$.\\
    $$\forall S, S \text{ is a set}, \forall \alpha, \beta, \gamma \in S$$\\

    \noindent An element $\alpha$ is equal to itself, this property is called reflexivity. If an element $\alpha$ is equal to an element $\beta$ then $\beta$ is also equal to $\alpha$, this property is called symmetry. If and element $\alpha$ is equivalent to an element $\beta$, and $\beta$ is also equivalent $\gamma$ then $\alpha$ is equivalent to $\gamma$, this property is called transitivity.
    \begin{align*}
      \text{(i) } & \alpha = \alpha, \text{(relexivity)}\\
      \text{(ii) } & \alpha = \beta \Rightarrow \beta = \alpha, \text{(symmetry)}\\
      \text{(iii) } & \alpha = \beta \land \beta = \gamma \Rightarrow \alpha = \gamma, \text{(transitivity)}
    \end{align*}
  }

  {\remark Well Defined Operation \stepcounter{tcount}\\\\
    \noindent An operation is said to be well defined if given the same inputs, the operation will always give the same outputs. Or symbolically,

    $$\forall \alpha, \beta \in S, f \text{ is an operation}, \alpha = \beta \Rightarrow f[\alpha] = f[\beta]$$

    \noindent at the moment we will take for granted the fact that the normal operations of numbers is well defined, that is addition, subtraction, multiplication, and division of 2 numbers.\\
  }

  {\example Isolate $x$ \stepcounter{tcount}\\\\
    \noindent We can apply the remark about well defined operations to manipulate our equations in the following manner.
    \begin{align}
      x + 2 & = 3\\
      x + 2 - 2 & = 3 - 2\\
      x & = 1
    \end{align}
  }

  {\remark Interpretation of Well Defined Operation in Terms of Equation Manipulation \stepcounter{tcount}\\
    For those that are no convinced by the manipulation above, or how the definition of a well defined operation is related to it. Another way to interperet a well defined operation is as an operation where if you apply the operation to a single number, you should always get the same result.\\

    \noindent In step $(1)$ we know that we have two equivalent objects $x + 2$ and $3$, the well defined operation that we apply to the two equivalent objects is "minus 2". Since minus is well defined, we conclude that $x + 2 - 2$ is equivalent to $3 - 2$, we collect the terms and conclude that $x$ is equivalent to $1$.
  }
  \pagebreak

  {\example Isolate $x$ \stepcounter{tcount}\\
    \begin{align*}
      2x & = 6\\
      \frac{1}{2} \cdot 2x & = \frac{1}{2} \cdot 6\\
      x & = 3
    \end{align*}
  }

  {\example Isolate $x$ \stepcounter{tcount}\\
    \begin{align*}
      \frac{x}{3} & = 4\\
      \frac{x}{3} \cdot 3 & = 4 \cdot 3\\
      x & = 12
    \end{align*}
  }

  {\example Isolate $y$ \stepcounter{tcount}\\
    \begin{align*}
      \frac{y + 2}{3} & = 4x\\
      3 \cdot \frac{y + 2}{3} & = 3 \cdot 4x\\
      y + 2 & = 12x\\
      y & = 12x - 2
    \end{align*}
  }

  {\example Isolate $y$ \stepcounter{tcount}\\
    \begin{align*}
      \frac{3 - 2y}{5} - y & = x\\
      \frac{3 - 2y}{5} & = x + y\\
      3 - 2y & = 5x + 5y\\
      3 - 5x & = 7y\\
      y = \frac{3 - 5x}{7}
    \end{align*}
  }

  {\remark Simplest Form \stepcounter{tcount}\\
    When we are dealing with equations with more than one variable, isolating one variable will result creating an equation in the form of a variable written in terms of another variable. However that form is not simple like a single number, there is nothing wrong with that. It just happens that the simpler form cannot contain all of the information of the original equation.
  }

  \pagebreak

  {\remark lorem ipsum tater tots

  }

\end{document}
